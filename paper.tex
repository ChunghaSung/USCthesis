\documentclass[11pt]{report}
\usepackage[
  dissertation
 ,final
 ,raggedbottom
]{USCthesis}

% guidelines for manuscript formatting: http://graduateschool.usc.edu/wp-content/themes/fictional-university-theme/assets/doc/Manuscript_Formatting_and_Documentation_Styles.pdf

%% our customizations %%%%%%%%%%%%%%%%%%%%%%%%%%%%%%%%%%%%%%%%%%%%%%%%%%
\usepackage[export]{adjustbox} % for frame option in \includegraphics
\usepackage{amsmath}
\usepackage{amssymb}
\usepackage{array}
\usepackage[utf8]{inputenc} % load inputenc before csquotes
\usepackage[english]{babel}
\usepackage[
  backend     = biber,
  doi         = true,
  hyperref    = true,
  maxbibnames = 99,
  sortlocale  = en_US,
  style       = numeric,
]{biblatex}
\usepackage{booktabs}
\usepackage{color, colortbl}
\usepackage{csquotes}
\usepackage{efbox}
\usepackage{enumitem}
\usepackage[shortcuts]{extdash} % use `\-/' to hyphenate words/phrases that have a dash in them
\usepackage[tt=false]{libertine} % libertine's \ttfamily isn't that great
\usepackage[T1]{fontenc} % load fonts before fontenc
\usepackage[symbol]{footmisc}
\usepackage[
  showframe = false,% draw a border around textwidth
  pass      = true, % force 8.5"x11" pagesize
]{geometry}
\usepackage{graphicx}
%\usepackage[notquote]{hanging} % enables negative indents in paragraphs
\usepackage{hyphenat}
\usepackage{ifthen}
\usepackage{lipsum}
\usepackage{multirow}
\usepackage{parnotes}
\usepackage{pdflscape} % rotate some pages in an {landscape} environment
\usepackage{pifont}
\usepackage{ragged2e}
\usepackage{seqsplit}
\usepackage{siunitx}
\usepackage{subcaption}
\usepackage{tabularx}
\usepackage{xcolor}
\usepackage{xspace}
\usepackage{url}

\usepackage[
  breaklinks    = true,
  colorlinks    = true,
  hypertexnames = false,
  pdfpagelabels = false,
  citecolor     = {blue!80!black},
  linkcolor     = {blue!80!black},
  urlcolor      = {blue!80!black},
]{hyperref} % load hyperref as the last package

% pkg: biblatex
\setlength\bibitemsep{0.5\baselineskip}                 % add a line between entries
\AtEveryBibitem{\iffieldundef{doi}{}{\clearfield{url}}} % if DOI, hide URL

\addbibresource{paper.bib}

% pkg: siunitx
% some guidelines https://physics.nist.gov/cuu/Units/checklist.html
\sisetup{
  tight-spacing  = true
  ,detect-family = true
  ,detect-mode   = true
  ,binary-units  = true    % support for MB, GB, etc.
  ,range-units   = single  % "3% to 5%" -> "3 to 5%"
  ,range-phrase  = --      % "3 to 5%"  -> "3--5%"
}

% pkg: babel, hyperref
\addto\extrasenglish{%
  \renewcommand{\chapterautorefname}{Chapter}
  \renewcommand{\sectionautorefname}{Section}
  \renewcommand{\subsectionautorefname}{Section}
  \renewcommand{\subsubsectionautorefname}{Section}
}

% pkg: url
\renewcommand{\UrlFont}{\footnotesize\tt}

% our custom commands
\renewcommand{\ttdefault}{cmtt} % use computer modern for teletype

%%% draft mode / toggle commands %%%
\usepackage{etoolbox}
\newtoggle{draft}
\settoggle{draft}{true} % change toggle for draft or final versions

\iftoggle{draft}{
  % if 'draft' toggle is true
  \overfullrule=10pt                       % highlight overfull hboxes
}{
  % if 'draft' toggle is false
  \PassOptionsToPackage{final}{showlabels} % hide labels on figures, etc
}

% if you're including existing papers into your thesis, it helps to put
% content behind a toggle (or conditional) so you only have to maintain
% and keep consistency on one copy. see "introduction.tex".
\newtoggle{thesis}
\settoggle{thesis}{true}

\usepackage[inline]{showlabels}
\renewcommand{\showlabelfont}{\sffamily \color{blue}}
\renewcommand{\showlabelsetlabel}[1]{\efbox{\showlabelfont #1}}
%%%%%%%%%%%%%%%%%%%%%%%%%%%%%%%%%%%%%%%%%%%%%%%%%%%%%%%%%%%%%%%%%%%%%%%%

%%% front matter %%%%%%%%%%%%%%%%%%%%%%%%%%%%%%%%%%%%%%%%%%%%%%%%%%%%%%%
\begin{document}

% title should be all caps
\title{[DISSERTATION TITLE]}

% use your full name!
% https://cs.stanford.edu/~knuth/news19.html
% "Let's celebrate everybody's full names"
\author{[Full Name]}

% major should be all caps
\majorfield{[MAJOR]}

% date should be May, August, or December (when degrees are conferred)
\submitdate{August 20XX}

%%% preface %%%%%%%%%%%%%%%%%%%%%%%%%%%%%%%%%%%%%%%%%%%%%%%%%%%%%%%%%%%%
\begin{preface}
  \prefacesection{Dedication}
  To everyone.


  \prefacesection{Acknowledgements}
  Thank you to everyone.


  {
  \hypersetup{hidelinks} % color all links black in the preface
  \tableofcontents
  \listoftables
  \listoffigures
  }

  \prefacesection{Abstract}
  \lipsum[2-3]

\end{preface}

%%% introduction %%%%%%%%%%%%%%%%%%%%%%%%%%%%%%%%%%%%%%%%%%%%%%%%%%%%%%%
\chapter{Introduction}
  \label{ch:introduction}

\graphicspath{}
\section{An example section}
  \label{sec:example}

\iftoggle{thesis}{%
  This is content that shows up only in the thesis.
}{
  Because of space constraints, we leave this as an exercise to the
  reader.
}

\iftoggle{draft}{%
  \textsf{[\textbf{comment}: Fix such and such, revise
  \autoref{sec:yet-another-example}.  Rewrite \autoref{ch:conclusions}.]}
}{}

\subsection{An example sub-section}
  \label{sec:another-example}

The end-to-end princple~\cite{10.1145/357401.357402}.

\subsubsection{An example sub-sub-section}
  \label{sec:yet-another-example}

This-is-an-overfull-hbox-WWWWWWWWWWWWWWWWWWWWWWWWWWWWWWWWWWW.

Overfull hboxes (text going into the margins) will have a visible marker
in the margin indicating where it's overfull.

\begin{table}
\centering
\begin{tabular}{lS}
\toprule
$x$      & \textbf{value} \\
\midrule
a        & 1.23           \\
b        & 3.456          \\
c        & 100.0002       \\
d        & 12345.0        \\
\bottomrule
\end{tabular}
\caption[short caption for the LOT]
        {longer caption for the paper}
\label{tbl:example}
\end{table}


%%% chapter %%%%%%%%%%%%%%%%%%%%%%%%%%%%%%%%%%%%%%%%%%%%%%%%%%%%%%%%%%%%

% if you're "stapling" together papers, it's easy to include your paper
% directory by way of symlinks, or copying the entire paper as a
% subdirectory.
%
% for example, if your paper directory looks like the following:
%
%   foobar/          - top level paper directory
%   foobar/fig/      - where all graphics and figures live
%   foobar/paper.bib - bibliography
%   foobar/paper.tex - monolithic .tex file for paper
%
% then you might use the folloiwng:
%
%   \graphicspath{foobar/fig}
%   \addbibresource{foobar/paper.bib}
%   \documentclass[11pt]{report}
\usepackage[
  dissertation
 ,final
 ,raggedbottom
]{USCthesis}

% guidelines for manuscript formatting: http://graduateschool.usc.edu/wp-content/themes/fictional-university-theme/assets/doc/Manuscript_Formatting_and_Documentation_Styles.pdf

%% our customizations %%%%%%%%%%%%%%%%%%%%%%%%%%%%%%%%%%%%%%%%%%%%%%%%%%
\usepackage[export]{adjustbox} % for frame option in \includegraphics
\usepackage{amsmath}
\usepackage{amssymb}
\usepackage{array}
\usepackage[utf8]{inputenc} % load inputenc before csquotes
\usepackage[english]{babel}
\usepackage[
  backend     = biber,
  doi         = true,
  hyperref    = true,
  maxbibnames = 99,
  sortlocale  = en_US,
  style       = numeric,
]{biblatex}
\usepackage{booktabs}
\usepackage{color, colortbl}
\usepackage{csquotes}
\usepackage{efbox}
\usepackage{enumitem}
\usepackage[shortcuts]{extdash} % use `\-/' to hyphenate words/phrases that have a dash in them
\usepackage[tt=false]{libertine} % libertine's \ttfamily isn't that great
\usepackage[T1]{fontenc} % load fonts before fontenc
\usepackage[symbol]{footmisc}
\usepackage[
  showframe = false,% draw a border around textwidth
  pass      = true, % force 8.5"x11" pagesize
]{geometry}
\usepackage{graphicx}
%\usepackage[notquote]{hanging} % enables negative indents in paragraphs
\usepackage{hyphenat}
\usepackage{ifthen}
\usepackage{lipsum}
\usepackage{multirow}
\usepackage{parnotes}
\usepackage{pdflscape} % rotate some pages in an {landscape} environment
\usepackage{pifont}
\usepackage{ragged2e}
\usepackage{seqsplit}
\usepackage{siunitx}
\usepackage{subcaption}
\usepackage{tabularx}
\usepackage{xcolor}
\usepackage{xspace}
\usepackage{url}

\usepackage[
  breaklinks    = true,
  colorlinks    = true,
  hypertexnames = false,
  pdfpagelabels = false,
  citecolor     = {blue!80!black},
  linkcolor     = {blue!80!black},
  urlcolor      = {blue!80!black},
]{hyperref} % load hyperref as the last package

% pkg: biblatex
\setlength\bibitemsep{0.5\baselineskip}                 % add a line between entries
\AtEveryBibitem{\iffieldundef{doi}{}{\clearfield{url}}} % if DOI, hide URL

\addbibresource{paper.bib}

% pkg: siunitx
% some guidelines https://physics.nist.gov/cuu/Units/checklist.html
\sisetup{
  tight-spacing  = true
  ,detect-family = true
  ,detect-mode   = true
  ,binary-units  = true    % support for MB, GB, etc.
  ,range-units   = single  % "3% to 5%" -> "3 to 5%"
  ,range-phrase  = --      % "3 to 5%"  -> "3--5%"
}

% pkg: babel, hyperref
\addto\extrasenglish{%
  \renewcommand{\chapterautorefname}{Chapter}
  \renewcommand{\sectionautorefname}{Section}
  \renewcommand{\subsectionautorefname}{Section}
  \renewcommand{\subsubsectionautorefname}{Section}
}

% pkg: url
\renewcommand{\UrlFont}{\footnotesize\tt}

% our custom commands
\renewcommand{\ttdefault}{cmtt} % use computer modern for teletype

%%% draft mode / toggle commands %%%
\usepackage{etoolbox}
\newtoggle{draft}
\settoggle{draft}{true} % change toggle for draft or final versions

\iftoggle{draft}{
  % if 'draft' toggle is true
  \overfullrule=10pt                       % highlight overfull hboxes
}{
  % if 'draft' toggle is false
  \PassOptionsToPackage{final}{showlabels} % hide labels on figures, etc
}

% if you're including existing papers into your thesis, it helps to put
% content behind a toggle (or conditional) so you only have to maintain
% and keep consistency on one copy. see "introduction.tex".
\newtoggle{thesis}
\settoggle{thesis}{true}

\usepackage[inline]{showlabels}
\renewcommand{\showlabelfont}{\sffamily \color{blue}}
\renewcommand{\showlabelsetlabel}[1]{\efbox{\showlabelfont #1}}
%%%%%%%%%%%%%%%%%%%%%%%%%%%%%%%%%%%%%%%%%%%%%%%%%%%%%%%%%%%%%%%%%%%%%%%%

%%% front matter %%%%%%%%%%%%%%%%%%%%%%%%%%%%%%%%%%%%%%%%%%%%%%%%%%%%%%%
\begin{document}

% title should be all caps
\title{[DISSERTATION TITLE]}

% use your full name!
% https://cs.stanford.edu/~knuth/news19.html
% "Let's celebrate everybody's full names"
\author{[Full Name]}

% major should be all caps
\majorfield{[MAJOR]}

% date should be May, August, or December (when degrees are conferred)
\submitdate{August 20XX}

%%% preface %%%%%%%%%%%%%%%%%%%%%%%%%%%%%%%%%%%%%%%%%%%%%%%%%%%%%%%%%%%%
\begin{preface}
  \prefacesection{Dedication}
  To everyone.


  \prefacesection{Acknowledgements}
  Thank you to everyone.


  {
  \hypersetup{hidelinks} % color all links black in the preface
  \tableofcontents
  \listoftables
  \listoffigures
  }

  \prefacesection{Abstract}
  \lipsum[2-3]

\end{preface}

%%% introduction %%%%%%%%%%%%%%%%%%%%%%%%%%%%%%%%%%%%%%%%%%%%%%%%%%%%%%%
\chapter{Introduction}
  \label{ch:introduction}

\graphicspath{}
\section{An example section}
  \label{sec:example}

\iftoggle{thesis}{%
  This is content that shows up only in the thesis.
}{
  Because of space constraints, we leave this as an exercise to the
  reader.
}

\iftoggle{draft}{%
  \textsf{[\textbf{comment}: Fix such and such, revise
  \autoref{sec:yet-another-example}.  Rewrite \autoref{ch:conclusions}.]}
}{}

\subsection{An example sub-section}
  \label{sec:another-example}

The end-to-end princple~\cite{10.1145/357401.357402}.

\subsubsection{An example sub-sub-section}
  \label{sec:yet-another-example}

This-is-an-overfull-hbox-WWWWWWWWWWWWWWWWWWWWWWWWWWWWWWWWWWW.

Overfull hboxes (text going into the margins) will have a visible marker
in the margin indicating where it's overfull.

\begin{table}
\centering
\begin{tabular}{lS}
\toprule
$x$      & \textbf{value} \\
\midrule
a        & 1.23           \\
b        & 3.456          \\
c        & 100.0002       \\
d        & 12345.0        \\
\bottomrule
\end{tabular}
\caption[short caption for the LOT]
        {longer caption for the paper}
\label{tbl:example}
\end{table}


%%% chapter %%%%%%%%%%%%%%%%%%%%%%%%%%%%%%%%%%%%%%%%%%%%%%%%%%%%%%%%%%%%

% if you're "stapling" together papers, it's easy to include your paper
% directory by way of symlinks, or copying the entire paper as a
% subdirectory.
%
% for example, if your paper directory looks like the following:
%
%   foobar/          - top level paper directory
%   foobar/fig/      - where all graphics and figures live
%   foobar/paper.bib - bibliography
%   foobar/paper.tex - monolithic .tex file for paper
%
% then you might use the folloiwng:
%
%   \graphicspath{foobar/fig}
%   \addbibresource{foobar/paper.bib}
%   \documentclass[11pt]{report}
\usepackage[
  dissertation
 ,final
 ,raggedbottom
]{USCthesis}

% guidelines for manuscript formatting: http://graduateschool.usc.edu/wp-content/themes/fictional-university-theme/assets/doc/Manuscript_Formatting_and_Documentation_Styles.pdf

%% our customizations %%%%%%%%%%%%%%%%%%%%%%%%%%%%%%%%%%%%%%%%%%%%%%%%%%
\usepackage[export]{adjustbox} % for frame option in \includegraphics
\usepackage{amsmath}
\usepackage{amssymb}
\usepackage{array}
\usepackage[utf8]{inputenc} % load inputenc before csquotes
\usepackage[english]{babel}
\usepackage[
  backend     = biber,
  doi         = true,
  hyperref    = true,
  maxbibnames = 99,
  sortlocale  = en_US,
  style       = numeric,
]{biblatex}
\usepackage{booktabs}
\usepackage{color, colortbl}
\usepackage{csquotes}
\usepackage{efbox}
\usepackage{enumitem}
\usepackage[shortcuts]{extdash} % use `\-/' to hyphenate words/phrases that have a dash in them
\usepackage[tt=false]{libertine} % libertine's \ttfamily isn't that great
\usepackage[T1]{fontenc} % load fonts before fontenc
\usepackage[symbol]{footmisc}
\usepackage[
  showframe = false,% draw a border around textwidth
  pass      = true, % force 8.5"x11" pagesize
]{geometry}
\usepackage{graphicx}
%\usepackage[notquote]{hanging} % enables negative indents in paragraphs
\usepackage{hyphenat}
\usepackage{ifthen}
\usepackage{lipsum}
\usepackage{multirow}
\usepackage{parnotes}
\usepackage{pdflscape} % rotate some pages in an {landscape} environment
\usepackage{pifont}
\usepackage{ragged2e}
\usepackage{seqsplit}
\usepackage{siunitx}
\usepackage{subcaption}
\usepackage{tabularx}
\usepackage{xcolor}
\usepackage{xspace}
\usepackage{url}

\usepackage[
  breaklinks    = true,
  colorlinks    = true,
  hypertexnames = false,
  pdfpagelabels = false,
  citecolor     = {blue!80!black},
  linkcolor     = {blue!80!black},
  urlcolor      = {blue!80!black},
]{hyperref} % load hyperref as the last package

% pkg: biblatex
\setlength\bibitemsep{0.5\baselineskip}                 % add a line between entries
\AtEveryBibitem{\iffieldundef{doi}{}{\clearfield{url}}} % if DOI, hide URL

\addbibresource{paper.bib}

% pkg: siunitx
% some guidelines https://physics.nist.gov/cuu/Units/checklist.html
\sisetup{
  tight-spacing  = true
  ,detect-family = true
  ,detect-mode   = true
  ,binary-units  = true    % support for MB, GB, etc.
  ,range-units   = single  % "3% to 5%" -> "3 to 5%"
  ,range-phrase  = --      % "3 to 5%"  -> "3--5%"
}

% pkg: babel, hyperref
\addto\extrasenglish{%
  \renewcommand{\chapterautorefname}{Chapter}
  \renewcommand{\sectionautorefname}{Section}
  \renewcommand{\subsectionautorefname}{Section}
  \renewcommand{\subsubsectionautorefname}{Section}
}

% pkg: url
\renewcommand{\UrlFont}{\footnotesize\tt}

% our custom commands
\renewcommand{\ttdefault}{cmtt} % use computer modern for teletype

%%% draft mode / toggle commands %%%
\usepackage{etoolbox}
\newtoggle{draft}
\settoggle{draft}{true} % change toggle for draft or final versions

\iftoggle{draft}{
  % if 'draft' toggle is true
  \overfullrule=10pt                       % highlight overfull hboxes
}{
  % if 'draft' toggle is false
  \PassOptionsToPackage{final}{showlabels} % hide labels on figures, etc
}

% if you're including existing papers into your thesis, it helps to put
% content behind a toggle (or conditional) so you only have to maintain
% and keep consistency on one copy. see "introduction.tex".
\newtoggle{thesis}
\settoggle{thesis}{true}

\usepackage[inline]{showlabels}
\renewcommand{\showlabelfont}{\sffamily \color{blue}}
\renewcommand{\showlabelsetlabel}[1]{\efbox{\showlabelfont #1}}
%%%%%%%%%%%%%%%%%%%%%%%%%%%%%%%%%%%%%%%%%%%%%%%%%%%%%%%%%%%%%%%%%%%%%%%%

%%% front matter %%%%%%%%%%%%%%%%%%%%%%%%%%%%%%%%%%%%%%%%%%%%%%%%%%%%%%%
\begin{document}

% title should be all caps
\title{[DISSERTATION TITLE]}

% use your full name!
% https://cs.stanford.edu/~knuth/news19.html
% "Let's celebrate everybody's full names"
\author{[Full Name]}

% major should be all caps
\majorfield{[MAJOR]}

% date should be May, August, or December (when degrees are conferred)
\submitdate{August 20XX}

%%% preface %%%%%%%%%%%%%%%%%%%%%%%%%%%%%%%%%%%%%%%%%%%%%%%%%%%%%%%%%%%%
\begin{preface}
  \prefacesection{Dedication}
  To everyone.


  \prefacesection{Acknowledgements}
  Thank you to everyone.


  {
  \hypersetup{hidelinks} % color all links black in the preface
  \tableofcontents
  \listoftables
  \listoffigures
  }

  \prefacesection{Abstract}
  \lipsum[2-3]

\end{preface}

%%% introduction %%%%%%%%%%%%%%%%%%%%%%%%%%%%%%%%%%%%%%%%%%%%%%%%%%%%%%%
\chapter{Introduction}
  \label{ch:introduction}

\graphicspath{}
\section{An example section}
  \label{sec:example}

\iftoggle{thesis}{%
  This is content that shows up only in the thesis.
}{
  Because of space constraints, we leave this as an exercise to the
  reader.
}

\iftoggle{draft}{%
  \textsf{[\textbf{comment}: Fix such and such, revise
  \autoref{sec:yet-another-example}.  Rewrite \autoref{ch:conclusions}.]}
}{}

\subsection{An example sub-section}
  \label{sec:another-example}

The end-to-end princple~\cite{10.1145/357401.357402}.

\subsubsection{An example sub-sub-section}
  \label{sec:yet-another-example}

This-is-an-overfull-hbox-WWWWWWWWWWWWWWWWWWWWWWWWWWWWWWWWWWW.

Overfull hboxes (text going into the margins) will have a visible marker
in the margin indicating where it's overfull.

\begin{table}
\centering
\begin{tabular}{lS}
\toprule
$x$      & \textbf{value} \\
\midrule
a        & 1.23           \\
b        & 3.456          \\
c        & 100.0002       \\
d        & 12345.0        \\
\bottomrule
\end{tabular}
\caption[short caption for the LOT]
        {longer caption for the paper}
\label{tbl:example}
\end{table}


%%% chapter %%%%%%%%%%%%%%%%%%%%%%%%%%%%%%%%%%%%%%%%%%%%%%%%%%%%%%%%%%%%

% if you're "stapling" together papers, it's easy to include your paper
% directory by way of symlinks, or copying the entire paper as a
% subdirectory.
%
% for example, if your paper directory looks like the following:
%
%   foobar/          - top level paper directory
%   foobar/fig/      - where all graphics and figures live
%   foobar/paper.bib - bibliography
%   foobar/paper.tex - monolithic .tex file for paper
%
% then you might use the folloiwng:
%
%   \graphicspath{foobar/fig}
%   \addbibresource{foobar/paper.bib}
%   \documentclass[11pt]{report}
\usepackage[
  dissertation
 ,final
 ,raggedbottom
]{USCthesis}

% guidelines for manuscript formatting: http://graduateschool.usc.edu/wp-content/themes/fictional-university-theme/assets/doc/Manuscript_Formatting_and_Documentation_Styles.pdf

%% our customizations %%%%%%%%%%%%%%%%%%%%%%%%%%%%%%%%%%%%%%%%%%%%%%%%%%
\usepackage[export]{adjustbox} % for frame option in \includegraphics
\usepackage{amsmath}
\usepackage{amssymb}
\usepackage{array}
\usepackage[utf8]{inputenc} % load inputenc before csquotes
\usepackage[english]{babel}
\usepackage[
  backend     = biber,
  doi         = true,
  hyperref    = true,
  maxbibnames = 99,
  sortlocale  = en_US,
  style       = numeric,
]{biblatex}
\usepackage{booktabs}
\usepackage{color, colortbl}
\usepackage{csquotes}
\usepackage{efbox}
\usepackage{enumitem}
\usepackage[shortcuts]{extdash} % use `\-/' to hyphenate words/phrases that have a dash in them
\usepackage[tt=false]{libertine} % libertine's \ttfamily isn't that great
\usepackage[T1]{fontenc} % load fonts before fontenc
\usepackage[symbol]{footmisc}
\usepackage[
  showframe = false,% draw a border around textwidth
  pass      = true, % force 8.5"x11" pagesize
]{geometry}
\usepackage{graphicx}
%\usepackage[notquote]{hanging} % enables negative indents in paragraphs
\usepackage{hyphenat}
\usepackage{ifthen}
\usepackage{lipsum}
\usepackage{multirow}
\usepackage{parnotes}
\usepackage{pdflscape} % rotate some pages in an {landscape} environment
\usepackage{pifont}
\usepackage{ragged2e}
\usepackage{seqsplit}
\usepackage{siunitx}
\usepackage{subcaption}
\usepackage{tabularx}
\usepackage{xcolor}
\usepackage{xspace}
\usepackage{url}

\usepackage[
  breaklinks    = true,
  colorlinks    = true,
  hypertexnames = false,
  pdfpagelabels = false,
  citecolor     = {blue!80!black},
  linkcolor     = {blue!80!black},
  urlcolor      = {blue!80!black},
]{hyperref} % load hyperref as the last package

% pkg: biblatex
\setlength\bibitemsep{0.5\baselineskip}                 % add a line between entries
\AtEveryBibitem{\iffieldundef{doi}{}{\clearfield{url}}} % if DOI, hide URL

\addbibresource{paper.bib}

% pkg: siunitx
% some guidelines https://physics.nist.gov/cuu/Units/checklist.html
\sisetup{
  tight-spacing  = true
  ,detect-family = true
  ,detect-mode   = true
  ,binary-units  = true    % support for MB, GB, etc.
  ,range-units   = single  % "3% to 5%" -> "3 to 5%"
  ,range-phrase  = --      % "3 to 5%"  -> "3--5%"
}

% pkg: babel, hyperref
\addto\extrasenglish{%
  \renewcommand{\chapterautorefname}{Chapter}
  \renewcommand{\sectionautorefname}{Section}
  \renewcommand{\subsectionautorefname}{Section}
  \renewcommand{\subsubsectionautorefname}{Section}
}

% pkg: url
\renewcommand{\UrlFont}{\footnotesize\tt}

% our custom commands
\renewcommand{\ttdefault}{cmtt} % use computer modern for teletype

%%% draft mode / toggle commands %%%
\usepackage{etoolbox}
\newtoggle{draft}
\settoggle{draft}{true} % change toggle for draft or final versions

\iftoggle{draft}{
  % if 'draft' toggle is true
  \overfullrule=10pt                       % highlight overfull hboxes
}{
  % if 'draft' toggle is false
  \PassOptionsToPackage{final}{showlabels} % hide labels on figures, etc
}

% if you're including existing papers into your thesis, it helps to put
% content behind a toggle (or conditional) so you only have to maintain
% and keep consistency on one copy. see "introduction.tex".
\newtoggle{thesis}
\settoggle{thesis}{true}

\usepackage[inline]{showlabels}
\renewcommand{\showlabelfont}{\sffamily \color{blue}}
\renewcommand{\showlabelsetlabel}[1]{\efbox{\showlabelfont #1}}
%%%%%%%%%%%%%%%%%%%%%%%%%%%%%%%%%%%%%%%%%%%%%%%%%%%%%%%%%%%%%%%%%%%%%%%%

%%% front matter %%%%%%%%%%%%%%%%%%%%%%%%%%%%%%%%%%%%%%%%%%%%%%%%%%%%%%%
\begin{document}

% title should be all caps
\title{[DISSERTATION TITLE]}

% use your full name!
% https://cs.stanford.edu/~knuth/news19.html
% "Let's celebrate everybody's full names"
\author{[Full Name]}

% major should be all caps
\majorfield{[MAJOR]}

% date should be May, August, or December (when degrees are conferred)
\submitdate{August 20XX}

%%% preface %%%%%%%%%%%%%%%%%%%%%%%%%%%%%%%%%%%%%%%%%%%%%%%%%%%%%%%%%%%%
\begin{preface}
  \prefacesection{Dedication}
  \input{dedication.tex}

  \prefacesection{Acknowledgements}
  \input{acknowledgements.tex}

  {
  \hypersetup{hidelinks} % color all links black in the preface
  \tableofcontents
  \listoftables
  \listoffigures
  }

  \prefacesection{Abstract}
  \input{abstract.tex}
\end{preface}

%%% introduction %%%%%%%%%%%%%%%%%%%%%%%%%%%%%%%%%%%%%%%%%%%%%%%%%%%%%%%
\chapter{Introduction}
  \label{ch:introduction}

\graphicspath{}
\input{introduction.tex}

%%% chapter %%%%%%%%%%%%%%%%%%%%%%%%%%%%%%%%%%%%%%%%%%%%%%%%%%%%%%%%%%%%

% if you're "stapling" together papers, it's easy to include your paper
% directory by way of symlinks, or copying the entire paper as a
% subdirectory.
%
% for example, if your paper directory looks like the following:
%
%   foobar/          - top level paper directory
%   foobar/fig/      - where all graphics and figures live
%   foobar/paper.bib - bibliography
%   foobar/paper.tex - monolithic .tex file for paper
%
% then you might use the folloiwng:
%
%   \graphicspath{foobar/fig}
%   \addbibresource{foobar/paper.bib}
%   \input{foobar/paper.tex}
%
% note that you'll have to modify the input file to make sure that the
% preamble (\documentclass, etc.) isn't included. to make your life
% easier, you could use some TeX conditionals to make it seamless.
%
% this requires some planning, but enables you to edit the individual
% paper and thesis chapter without tracking and porting changes between
% multiple directories and repositories:
%
% for example, at the beginning of foobar/paper.tex (before
% \documentclass):
%
%   \newif\ifdissertation
%   \dissertationtrue      % (or \dissertationfalse for the standalone)
%
%   \ifdissertation
%   \else
%   \documentclass...
%   \fi
%
%   \ifdissertation
%   \else
%   \begin{document}
%   \fi
%
%   [...paper content here...]
%
%   \ifdissertation
%   \else
%   \end{document}
%   \fi

%%% conclusions %%%%%%%%%%%%%%%%%%%%%%%%%%%%%%%%%%%%%%%%%%%%%%%%%%%%%%%%
\chapter{Conclusions}
  \label{ch:conclusions}

\graphicspath{}
\input{conclusions}

%%% bibliography %%%%%%%%%%%%%%%%%%%%%%%%%%%%%%%%%%%%%%%%%%%%%%%%%%%%%%%
%
%  \printbibliography in biblatex is great, but doesn't allow for the
%  greatest customization, so we'll use the package biblatex + biber
%  backend to meet some requirements:
%
%  * bibliography should be an un-numbered chapter, and still have a
%    pdfbookmark and a line in the table of contents
%
%  * bibliography contents should be singlespace, and optionally a smaller
%    font
%
%  * first line of this "chapter" should be in the same spot as the first
%    line of preface sections (e.g., acknowledgement)
%
%  * we use \raggedright so things like URLs and DOIs aren't stretched out.
%
\clearpage
\chapter*{Bibliography}
\addcontentsline{toc}{chapter}{Bibliography}

\begin{singlespace}
  % increase penalty such that we don't break entries over pages
  % source: https://tex.stackexchange.com/a/43275
  \patchcmd{\bibsetup}{\interlinepenalty=5000}{\interlinepenalty=10000}{}{}

  % reduce spacing between each bibentry
  \setlength\bibitemsep{0.9\baselineskip}

  % don't justify-align entries: this prevents stretching out each line
  \raggedright
  \printbibliography[
    heading = none
  ]
\end{singlespace}

\end{document}

%
% note that you'll have to modify the input file to make sure that the
% preamble (\documentclass, etc.) isn't included. to make your life
% easier, you could use some TeX conditionals to make it seamless.
%
% this requires some planning, but enables you to edit the individual
% paper and thesis chapter without tracking and porting changes between
% multiple directories and repositories:
%
% for example, at the beginning of foobar/paper.tex (before
% \documentclass):
%
%   \newif\ifdissertation
%   \dissertationtrue      % (or \dissertationfalse for the standalone)
%
%   \ifdissertation
%   \else
%   \documentclass...
%   \fi
%
%   \ifdissertation
%   \else
%   \begin{document}
%   \fi
%
%   [...paper content here...]
%
%   \ifdissertation
%   \else
%   \end{document}
%   \fi

%%% conclusions %%%%%%%%%%%%%%%%%%%%%%%%%%%%%%%%%%%%%%%%%%%%%%%%%%%%%%%%
\chapter{Conclusions}
  \label{ch:conclusions}

\graphicspath{}
\lipsum[5-4]


%%% bibliography %%%%%%%%%%%%%%%%%%%%%%%%%%%%%%%%%%%%%%%%%%%%%%%%%%%%%%%
%
%  \printbibliography in biblatex is great, but doesn't allow for the
%  greatest customization, so we'll use the package biblatex + biber
%  backend to meet some requirements:
%
%  * bibliography should be an un-numbered chapter, and still have a
%    pdfbookmark and a line in the table of contents
%
%  * bibliography contents should be singlespace, and optionally a smaller
%    font
%
%  * first line of this "chapter" should be in the same spot as the first
%    line of preface sections (e.g., acknowledgement)
%
%  * we use \raggedright so things like URLs and DOIs aren't stretched out.
%
\clearpage
\chapter*{Bibliography}
\addcontentsline{toc}{chapter}{Bibliography}

\begin{singlespace}
  % increase penalty such that we don't break entries over pages
  % source: https://tex.stackexchange.com/a/43275
  \patchcmd{\bibsetup}{\interlinepenalty=5000}{\interlinepenalty=10000}{}{}

  % reduce spacing between each bibentry
  \setlength\bibitemsep{0.9\baselineskip}

  % don't justify-align entries: this prevents stretching out each line
  \raggedright
  \printbibliography[
    heading = none
  ]
\end{singlespace}

\end{document}

%
% note that you'll have to modify the input file to make sure that the
% preamble (\documentclass, etc.) isn't included. to make your life
% easier, you could use some TeX conditionals to make it seamless.
%
% this requires some planning, but enables you to edit the individual
% paper and thesis chapter without tracking and porting changes between
% multiple directories and repositories:
%
% for example, at the beginning of foobar/paper.tex (before
% \documentclass):
%
%   \newif\ifdissertation
%   \dissertationtrue      % (or \dissertationfalse for the standalone)
%
%   \ifdissertation
%   \else
%   \documentclass...
%   \fi
%
%   \ifdissertation
%   \else
%   \begin{document}
%   \fi
%
%   [...paper content here...]
%
%   \ifdissertation
%   \else
%   \end{document}
%   \fi

%%% conclusions %%%%%%%%%%%%%%%%%%%%%%%%%%%%%%%%%%%%%%%%%%%%%%%%%%%%%%%%
\chapter{Conclusions}
  \label{ch:conclusions}

\graphicspath{}
\lipsum[5-4]


%%% bibliography %%%%%%%%%%%%%%%%%%%%%%%%%%%%%%%%%%%%%%%%%%%%%%%%%%%%%%%
%
%  \printbibliography in biblatex is great, but doesn't allow for the
%  greatest customization, so we'll use the package biblatex + biber
%  backend to meet some requirements:
%
%  * bibliography should be an un-numbered chapter, and still have a
%    pdfbookmark and a line in the table of contents
%
%  * bibliography contents should be singlespace, and optionally a smaller
%    font
%
%  * first line of this "chapter" should be in the same spot as the first
%    line of preface sections (e.g., acknowledgement)
%
%  * we use \raggedright so things like URLs and DOIs aren't stretched out.
%
\clearpage
\chapter*{Bibliography}
\addcontentsline{toc}{chapter}{Bibliography}

\begin{singlespace}
  % increase penalty such that we don't break entries over pages
  % source: https://tex.stackexchange.com/a/43275
  \patchcmd{\bibsetup}{\interlinepenalty=5000}{\interlinepenalty=10000}{}{}

  % reduce spacing between each bibentry
  \setlength\bibitemsep{0.9\baselineskip}

  % don't justify-align entries: this prevents stretching out each line
  \raggedright
  \printbibliography[
    heading = none
  ]
\end{singlespace}

\end{document}

%
% note that you'll have to modify the input file to make sure that the
% preamble (\documentclass, etc.) isn't included. to make your life
% easier, you could use some TeX conditionals to make it seamless.
%
% this requires some planning, but enables you to edit the individual
% paper and thesis chapter without tracking and porting changes between
% multiple directories and repositories:
%
% for example, at the beginning of foobar/paper.tex (before
% \documentclass):
%
%   \newif\ifdissertation
%   \dissertationtrue      % (or \dissertationfalse for the standalone)
%
%   \ifdissertation
%   \else
%   \documentclass...
%   \fi
%
%   \ifdissertation
%   \else
%   \begin{document}
%   \fi
%
%   [...paper content here...]
%
%   \ifdissertation
%   \else
%   \end{document}
%   \fi

%%% conclusions %%%%%%%%%%%%%%%%%%%%%%%%%%%%%%%%%%%%%%%%%%%%%%%%%%%%%%%%
\chapter{Conclusions}
  \label{ch:conclusions}

\graphicspath{}
\lipsum[5-4]


%%% bibliography %%%%%%%%%%%%%%%%%%%%%%%%%%%%%%%%%%%%%%%%%%%%%%%%%%%%%%%
%
%  \printbibliography in biblatex is great, but doesn't allow for the
%  greatest customization, so we'll use the package biblatex + biber
%  backend to meet some requirements:
%
%  * bibliography should be an un-numbered chapter, and still have a
%    pdfbookmark and a line in the table of contents
%
%  * bibliography contents should be singlespace, and optionally a smaller
%    font
%
%  * first line of this "chapter" should be in the same spot as the first
%    line of preface sections (e.g., acknowledgement)
%
%  * we use \raggedright so things like URLs and DOIs aren't stretched out.
%
\clearpage
\chapter*{Bibliography}
\addcontentsline{toc}{chapter}{Bibliography}

\begin{singlespace}
  % increase penalty such that we don't break entries over pages
  % source: https://tex.stackexchange.com/a/43275
  \patchcmd{\bibsetup}{\interlinepenalty=5000}{\interlinepenalty=10000}{}{}

  % reduce spacing between each bibentry
  \setlength\bibitemsep{0.9\baselineskip}

  % don't justify-align entries: this prevents stretching out each line
  \raggedright
  \printbibliography[
    heading = none
  ]
\end{singlespace}

\end{document}
